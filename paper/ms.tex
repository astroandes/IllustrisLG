\documentclass{emulateapj}
\submitted{{\it Submitted for publication in ApJ}}
\usepackage{multirow,color,wrapfig,ulem}
\usepackage {graphicx}
\usepackage{graphics}
\usepackage{amsmath}
\usepackage[dvips]{epsfig}
\bibliographystyle{apj}
\newcommand{\avg}[1]{\langle{#1}\rangle}  
\newcommand{\nscatt}{\langle N_{\rm  scatt}\rangle}
\newcommand{\ly}{{\ifmmode{{\rm Ly}\alpha~}\else{Ly$\alpha$~}\fi}}
\newcommand{\hMpc}{{\ifmmode{h^{-1}{\rm Mpc}}\else{$h^{-1}$Mpc }\fi}}   
\newcommand{\hGpc}{{\ifmmode{h^{-1}{\rm Gpc}}\else{$h^{-1}$Gpc }\fi}}   
\newcommand{\hmpc}{{\ifmmode{h^{-1}{\rm Mpc}}\else{$h^{-1}$Mpc }\fi}}  
\newcommand{\hkpc}{{\ifmmode{h^{-1}{\rm kpc}}\else{$h^{-1}$kpc }\fi}}  
\newcommand{\hMsun}{{\ifmmode{h^{-1}{\rm
        {M_{\odot}}}}\else{$h^{-1}{\rm{M_{\odot}}}$}\fi}}   
\newcommand{\hmsun}{{\ifmmode{h^{-1}{\rm
        {M_{\odot}}}}\else{$h^{-1}{\rm{M_{\odot}}}$}\fi}}   
\newcommand{\Msun}{{\ifmmode{{\rm {M_{\odot}}}}\else{${\rm{M_{\odot}}}$}\fi}}  
\newcommand{\msun}{{\ifmmode{{\rm {M_{\odot}}}}\else{${\rm{M_{\odot}}}$}\fi}}  
\newcommand{\lya}{{Lyman $\alpha$~}}
\newcommand{\clara}{{\texttt{CLARA}}~}
\newcommand{\rand}{{\ifmmode{{\mathcal{R}}}\else{${\mathcal{R}}$ }\fi}}  
\newcommand{\hs}{{\hspace{1mm}}}  
\newcommand{\kms}{{\ifmmode{{\mathrm{\,km\ s}^{-1}}}\else{\,km~s$^{-1}$}\fi}}
% definition to produce a "less than or similar to" symbol
\def\lsim{~\rlap{$<$}{\lower 1.0ex\hbox{$\sim$}}}
% definition to produce a "greater than or similar to" symbol
\def\gsim{~\rlap{$>$}{\lower 1.0ex\hbox{$\sim$}}}
%@arxiver{fig3.pdf,fig11a.pdf, fig11b.pdf} 
\begin{document}



\title{The Local Group in the Illustris Simulation}
\shorttitle{The LG in Illustris}

\shortauthors{Forero-Romero}

\author{Jaime E. Forero-Romero}  
\affil{Departamento de F\'{i}sica, Universidad de los Andes, Cra. 1
No. 18A-10, Edificio Ip, Bogot\'a, Colombia}
\email{je.forero@uniandes.edu.co}

\keywords{methods: numerical} 
\begin{abstract}
We present the properties of systems ressembling the Local Group (LG)
at redshift $z=0$ in the Illustris simulation. 
We find 51 LG-like systems in the simulation volume.
\end{abstract}


\section{Introduction}
\label{sec:intro}

\cite{Illustris}



\section{Simulation Description}

\section{Observational Constraints}  

\section{Results}

\subsection{Dark Matter Content}

\subsection{Baryonic Budget and morphology}

\subsection{Star formation rate and magnitudes}

\subsection{Kinematics}

\subsection{Galaxy Satellites}

\subsection{Alignments with the Cosmic Web}


\section{Discussion}




\bibliography{references}

\end{document}

